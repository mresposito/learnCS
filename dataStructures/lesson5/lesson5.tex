\documentclass[../notebook.tex]{subfiles} 

\linespread{1.2} % line spread
\begin{document}
\begin{center}

\end{center}

\section{Abstract Data Type} % (fold)
\label{sec:Abstact Data Type}
An Abstract Data Type (ADT) is a model for data structures. What it allows us to do, is to focus on the structure itself instead of the implementation. This is useful, because like this we can also look at different implementations of the same structures. \\
Almost every ADT that we will be looking at, will be an implementation of a {\bf dictionary}. That is, we will have a pair $key$, $value$ insert inside the data structure. Then, we will be retrieving the values by looking up the associated $key$. A dictionary is very useful for several reasons. 
\begin{enumerate}
  \item Allows you abstract between key and value.
  \item Allows many operations on data, such as sorting keys of finding min/max keys.
  \item Allows faster insert/retrieval.
\end{enumerate}
% section Abstact Data Type (end)
\section{ Linked Lists } % (fold)
\label{sec:Stacks and queues}
The first ADT that we will be looking at are Linked Lists. 
% section Stacks and queues (end)
\end{document}
